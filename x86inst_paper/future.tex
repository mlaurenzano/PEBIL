\section{Future Work}

Despite some success in terms of efficiency, there are several more techniques
that might make the instrumented code even more efficient. Because we are relocating
the text to give ourselves as much space as possible, rather than inserting just a branch
that transfers control to the instrumentation code we have the opportunity to inline
the instrumentation code itsself in order to reduce  or eliminate the control interruptions
that otherwise must be taken when inserting instrumentation code.

Currently we save all general purpose registers around each function call and we allow the
tool developer to state which resgiters are saved around instrumentation snippets. For instrumentation
snippets we could automatically detect which registers are killed by the snippet and live at the entry point
of the snippet adn automatically save only those. Similarly we could perform register analysis in order to determine whether there is
state that doesn't need to be saved around instrumentation functions. And similar to Pin, we could perform
liveness on the bits of the eflags/rflags register to determine whether it must be saved and
restored at each instrumentation point. Saving and restoring state is a large portion of the overhead associated with performing
small tasks in instrumentation snippets and these optimizations can help reduce or eliminate these overheads
entirely.

