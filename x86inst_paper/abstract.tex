\begin{it}
Dynamic binary instrumentation toolkits that are in use today
produce instrumented code that is at a performance disadvantage to the instrumented code
produced by static binary instrumentation toolkits because dynamic binary instrumentation toolkits 
act at runtime. Therefore in cases where efficiency is paramount it is important to have a
binary instrumentation toolkit capable of meeting that need.

In this work we present X86ElfInstrumentor, a static binary instrumentation
toolkit for Linux on x86/x86\_64 platforms that uses wholesale
code relocation in order to remedy the difficulty created by the platforms' use
of variable-length instructions. Code relocation of this kind allows the
instrumentation tool to reorganize the application code in such a way that it
can use the fast but far-reaching constructs to transfer control
from the application to the instrumentation code rather than relying on multiple
jumps or interrupts for the transfer. Furthermore, the API includes a means of
allowing the tool developer to insert hand-coded assembly in a very lightweight
way rather than relying solely on the insertion of entire instrumentation functions.
These techniques yield very efficient instrumentation tools, with overheads 
for basic block counting that are an
average of 48\% of the overhead imposed by Pin, 18\% of the overhead imposed by
DynamoRIO, 10\% of the overhead of Valgrind, and 5\% of the overhead of Dyninst.
\end{it}
