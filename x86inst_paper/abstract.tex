\begin{it}

Binary instrumentation facilitates the insertion of additional code into an
executable in order to observe or modify the behavior of application runs. There
are two main approaches to binary instrumentation: static and dynamic binary
instrumentation. In this paper we present PEBIL, an efficient static binary
instrumentation toolkit for Linux on x86/x86\_64 platforms. PEBIL is similar to
other toolkits in terms of how additional code is inserted. However, it is
unique because it uses function-level code relocation in order to remedy the
difficulty created by the underlying variable-length instruction set. Code
relocation of this kind allows the reorganization of the application code in
such a way that it can use fast far-reaching constructs to transfer control from
the application to the instrumentation code. Furthermore, the PEBIL API provides
tool developers means to insert lightweight hand-coded assembly rather than
relying solely on the insertion of instrumentation functions. These features
enable the implementation of efficient instrumentation tools with PEBIL. The
overhead introduced by a simple basic block counting tool created using PEBIL is
an average of 1.6x less than the overhead introduced by Pin, 4.7x less than the
overhead of DynamoRIO, 7.8x less than the overhead of Valgrind, and 23x less
than the overhead of Dyninst.

\end{it}
